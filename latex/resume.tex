%!TEX TS-program = xelatex
%!TEX encoding = UTF-8 Unicode
% Awesome CV LaTeX Template
%
% This template has been downloaded from:
% https://github.com/posquit0/Awesome-CV
%
% Author:
% Claud D. Park <posquit0.bj@gmail.com>
% http://www.posquit0.com
%
% Template license:
% CC BY-SA 4.0 (https://creativecommons.org/licenses/by-sa/4.0/)
%


%%%%%%%%%%%%%%%%%%%%%%%%%%%%%%%%%%%%%%
%     Configuration
%%%%%%%%%%%%%%%%%%%%%%%%%%%%%%%%%%%%%%
%%% Themes: Awesome-CV
\documentclass[]{awesome-cv}
\usepackage{textcomp}
\usepackage{multicol}
%%% Override a directory location for fonts(default: 'fonts/')
\fontdir[fonts/]

%%% Configure a directory location for sections
\newcommand*{\sectiondir}{resume/}

%%% Override color
% Awesome Colors: awesome-emerald, awesome-skyblue, awesome-red, awesome-pink, awesome-orange
%                 awesome-nephritis, awesome-concrete, awesome-darknight
%% Color for highlight
% Define your custom color if you don't like awesome colors
\colorlet{awesome}{awesome-red}
%\definecolor{awesome}{HTML}{CA63A8}
%% Colors for text
%\definecolor{darktext}{HTML}{414141}
%\definecolor{text}{HTML}{414141}
%\definecolor{graytext}{HTML}{414141}
%\definecolor{lighttext}{HTML}{414141}

%%% Override a separator for social informations in header(default: ' | ')
%\headersocialsep[\quad\textbar\quad]
    \begin{document}
    
%%%%%%%%%%%%%%%%%%%%%%%%%%%%%%%%%%%%%%
%     Profile
%%%%%%%%%%%%%%%%%%%%%%%%%%%%%%%%%%%%%%
\begin{center}
	\headerfirstnamestyle{Nicholas} \headerlastnamestyle{Hubbard} \\
	\vspace{2mm}
	\bodyfont{\faEnvelope\ hubbardn95@gmail.com} | {\faMobile\ (586) 383-3499} | {\faMapMarker\ Metro Detroit Area, MI} | {\faLink\ https://github.com/nhubbard}
\end{center}
%%%%%%%%%%%%%%%%%%%%%%%%%%%%%%%%%%%%%%
%     Education
%%%%%%%%%%%%%%%%%%%%%%%%%%%%%%%%%%%%%%
\cvsection{Education}
\begin{cventries}
	\cventry
	{BS in Computer Science/Software Engineering (Dual Major)}
	{Gannon University}
	{Erie, PA}
	{August 2020 – Present}
	{GPA: 3.1}
\end{cventries}

\vspace{-2mm}
%%%%%%%%%%%%%%%%%%%%%%%%%%%%%%%%%%%%%%
%     Experience
%%%%%%%%%%%%%%%%%%%%%%%%%%%%%%%%%%%%%%
\cvsection{Work Experience}
\begin{cventries}
	\cventry
	{Infrastructure Technician (Work Study)}
	{Gannon University}
	{Erie, PA}
	{October 2021 – Present}
	{\begin{cvitems}
		\item {Maintain, backup, update and set up CIS department servers, networking, and monitoring}
		\item {Deploy, update, and maintain shared department computer labs (80+ desktops)}
		\item {Completed major server room power delivery project under budget and over specification}
		\end{cvitems}}
	\cventry
	{Systems Engineering Intern}
	{Rock Central (Rocket Mortgage)}
	{Detroit, MI and Remote}
	{May 2021 – August 2021}
	{\begin{cvitems}
		\item {Migrated 60+ wiki pages from JIRA to MkDocs and GitHub (Markdown)}
		\item {Rewrote many pages to match updated and current practices}
		\item {Wrote multiple Python plugins for MkDocs to enable custom functionality lost during JIRA migration, each with full documentation and 100\% test coverage}
		\item {Redesigned MkDocs platform around the MkDocs for Material theme}
		\item {Completed entire project within the duration of the internship, with team members commending the final product}
		\end{cvitems}}
\end{cventries}
\cvsection{Skills}
\begin{cventries}
	\bodyfont{
	\textbf{Programming Languages}\\
	\begin{multicols}{3}
		\begin{itemize}
			\item{Java 11+}
			\item{Kotlin}
			\item{Kotlin Coroutines}
			\item{Kotlin Serialization}
			\item{Kotlin IO}
			\item{Gradle (Groovy and Kotlin DSLs)}
			\item{Python 3}
			\item{Modern Android Development (MAD)}
			\item{HTML 5}
			\item{CSS 3}
			\item{Tailwind CSS v3}
			\item{JavaScript (ES6+)}
			\item{TypeScript}
			\item{MySQL}
			\item{PostgreSQL}
		\end{itemize}
	\end{multicols}
	\textbf{Markup and Typesetting}\\
	\begin{itemize}
		\item{Markdown (including GFM and other extensions)}
		\item{LaTeX (this resume was created with LaTeX!)}
	\end{itemize}
	}
\end{cventries}

%\vspace{-7mm}
\cvsection{Projects}
\begin{cventries}
	\cventry
	{An automatic converter to migrate pure CSS projects to Tailwind CSS. In progress.}
	{Autoclave}
	{Kotlin, Kotlin Coroutines, JUnit 5, Gradle Kotlin DSL, CSS 3, Tailwind CSS v3}
	{(currently private and in development, can demo on request)}
	{}
	
	\vspace{-5mm}
	\cventry
	{An easy-to-use vision toolkit for working with SSD and SSD-Lite TensorFlow object detection models.}
	{VTK}
	{Python 3, TensorFlow, NVIDIA CUDA, Machine learning, OpenCV}
	{https://github.com/Robocubs/vtk}
	{}
	
	\vspace{-5mm}
	\cventry
	{An end-to-end machine learning model inference tool.}
	{MLRun}
	{Python 3, Numpy, NVIDIA CUDA, NVIDIA Jetson Platform(s), Google Coral AI Accelerators, OpenCV, FIRST Robotics NetworkTables}
	{https://github.com/Robocubs/mlrun}
	{}
	
	\vspace{-5mm}
	\cventry
	{A mobile app for Blackbaud onCampus-based LMS systems.}
	{Librecampus}
	{Kotlin, Kotlin Coroutines, Selenium, JUnit 5, OkHttp, Retrofit 2, Modern Android Development (MAD), Firebase}
	{(private, can show upon request; no demo can be provided)}
	{}
	
	\vspace{-5mm}
	\cventry
	{Stock sentiment analysis using Python, Goldman Sachs Marquee, TensorFlow, and NewsAPI.}
	{StockWise}
	{Python 3, Goldman Sachs Marquee, TensorFlow, NewsAPI, VADER Sentiment Analysis, Bootstrap, Django 2}
	{https://devpost.com/software/mhacks11-ypgx83}
	{}
	
	\vspace{-5mm}
\end{cventries}
\cvsection{Awards}
\begin{cvhonors}
	\cvhonor
	{FIRST Robotics Autonomous Award}
	{Recognized for creating the first machine learning-based vision system in FIRST Robotics, used to detect and track objects on the game field. The code for the project was later released under the name VTK.}
	{FIRST In Michigan Alpina 2 Event}
	{March 2019}
\end{cvhonors}
\ 
\end{document}